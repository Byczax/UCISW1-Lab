%!TEX program = xelatex
%Template created by: Maciej Byczko
\documentclass[a4paper,12pt]{extarticle}  %typ dokumentu

% \usepackage[utf8]{inputenc} %rodzaj czcionki w dokumencie
\usepackage{geometry} %poprawienie marginesów
\usepackage{polski} %polskie znaki
\usepackage{multirow} %tabela
\usepackage{graphicx} %tabela
\usepackage{float} %poprawienie pozycji
\usepackage{fancyhdr} % header i footer
\usepackage{karnaugh-map} % rysowanie siatek karnaugh
\usepackage{hyperref} %tworzenie odnośników, \url{<url>}, \href{<file path, link>}{<text with link>} \pageref{}
\usepackage{amsmath} % Matma
\usepackage{boldline}%edytowanie grubości krawędzi w tabelach \hlineB{} \clineB{}{}
\usepackage{array}%grubsze kolumny w tabeli
\usepackage{bigstrut}
\usepackage{caption}
\usepackage{listings} %pisanie kodu w ładny sposób, begin{listings}[language=<język>]...end{listings} tak samo jak nazwa paczki
\usepackage{subcaption}

%Ustawienie paczki hyperref
\hypersetup{
     colorlinks,
     citecolor=black,
     filecolor=black,
     linkcolor=black,
     urlcolor=black
}

\definecolor{backcolour}{rgb}{0.95,0.95,0.92}
\definecolor{AO}{rgb}{0,0.5,0}
\definecolor{ZeroBlue}{rgb}{0,0.28,0.73}
\definecolor{DarkRed}{rgb}{0.85,0.16,0.16}

\lstset{
breaklines=true,
language=vhdl,
numbers=left,
tabsize=2,
numberstyle=\tiny,
backgroundcolor=\color{backcolour},
breakatwhitespace=false,
showspaces=false,                
showstringspaces=false,
showtabs=false,
commentstyle=\color{gray},
keywordstyle=\color{ZeroBlue},
% keywordstyle={[2]\color{DarkRed}},
% keywordstyle={[3]\color{ZeroBlue}},
}
\graphicspath{{pictures/}}
\geometry{margin=0.7in}
\pagestyle{fancy}

\cfoot{Strona \thepage}
\rhead{Strona \thepage}
\lhead{\typdoc}
\newcolumntype{?}{!{\vrule width 1.5pt}}

\title{\tytul}
\author{\tworcy}
\date{\data}

%-----------------------PRZYDATNE LINKI----------------------------------
%link do tworzenia tabeli https://tablesgenerator.com
%symbole matematyczne: https://oeis.org/wiki/List_of_LaTeX_mathematical_symbols
%narzędzia matematyczne: https://en.wikibooks.org/wiki/LaTeX/Mathematics
%krótkie podpowiedzi http://www.mif.pg.gda.pl/homepages/sylas/students/wdi/doc/latex-sciaga.html
%symbole do schematów: http://texdoc.net/texmf-dist/doc/latex/circuitikz/circuitikzmanual.pdf
%----------------------------------------------------------------------

%-----------------------SEKCJA DANYCH----------------------------------
\def\tytul{Układy Kombinacyjne i Sekwencyjne w VHDL-u} %<<< tytuł ćwiczenia
\def\nrcw{5} %<<< numer ćwiczenia
\def\data{6 Grudnia 2021r.} %<< data wykonania
\def\prowadzacy{dr inż. Jacek Mazurkiewicz} %<<<prowadzący
\def\nrgrupy{B} %<<<numer grupy
\def\tworcy{Maciej Byczko\\Bartosz Matysiak} %<<< autorzy
\def\zajinfo{PN 10:50 TP} %<<< informacje dotyczące zajęć
\def\typdoc{Sprawozdanie} %<<< typ dokumentu tj Sprawozdanie, zadania itp. {Matematyka dyskretna/Sprawozdanie z Miernictwa}
\begin{document}
\setlength{\headheight}{15pt}

\newcommand{\ov}[1]{\overline{#1} \ }

%-------------------------------------TABELA-DANYCH--------------------------------------------------
\begin{table}[H]
	\centering
	\resizebox{\textwidth}{!}{
		\begin{tabular}{|c|c|c|}\hline
			\begin{tabular}[c]{@{}c@{}}                     \tworcy     \end{tabular} &
			\begin{tabular}[c]{@{}c@{}}Prowadzący:\\        \prowadzacy \end{tabular} &
			\begin{tabular}[c]{@{}c@{}}Numer ćwiczenia\\    \nrcw       \end{tabular}          \\ \hline
			\begin{tabular}[c]{@{}c@{}}                     \zajinfo    \end{tabular} &
			\begin{tabular}[c]{@{}c@{}}Temat ćwiczenia:\\   \tytul      \end{tabular} & Ocena: \\ \hline
			\begin{tabular}[c]{@{}c@{}}Grupa:\\          \nrgrupy    \end{tabular}    &
			\begin{tabular}[c]{@{}c@{}}Data wykonania:\\    \data       \end{tabular} &        \\ \hline
		\end{tabular}}
\end{table}
%----------------------------------------------------------------------------------------------------
\tableofcontents
\cleardoublepage
\section{Zadanie 1}
\subsection{Polecenie}
Implementacja funkcji logicznej \textbf{$G(w,x,y,z) = \prod(0, 2, 3, 4, 6, 7, 9, 11, 12, 13, 15)$} w VHDL-u za pomocą:
\begin{enumerate}
	\item Zapis równań boolowskich
	\item Metoda zapisu tablicowego
\end{enumerate}
\subsection{Rozwiązanie}
% \subsubsection{Schemat stanów}
\subsubsection{Tabela prawdy}
\begin{table}[H]
	\centering
	\resizebox{0.5\textwidth}{!}{
		\begin{tabular}{?c?c|c|c|c?c?}\hlineB{2.5}
			Kod dziesiętny & w & x & y & z & G \bigstrut \\\hlineB{2.5}
			0              & 0 & 0 & 0 & 0 & 0 \bigstrut \\\hline
			1              & 0 & 0 & 0 & 1 & 1 \bigstrut \\\hline
			2              & 0 & 0 & 1 & 0 & 0 \bigstrut \\\hline
			3              & 0 & 0 & 1 & 1 & 0 \bigstrut \\\hline
			4              & 0 & 1 & 0 & 0 & 0 \bigstrut \\\hline
			5              & 0 & 1 & 0 & 1 & 1 \bigstrut \\\hline
			6              & 0 & 1 & 1 & 0 & 0 \bigstrut \\\hline
			7              & 0 & 1 & 1 & 1 & 0 \bigstrut \\\hline
			8              & 1 & 0 & 0 & 0 & 1 \bigstrut \\\hline
			9              & 1 & 0 & 0 & 1 & 0 \bigstrut \\\hline
			10             & 1 & 0 & 1 & 0 & 1 \bigstrut \\\hline
			11             & 1 & 0 & 1 & 1 & 0 \bigstrut \\\hline
			12             & 1 & 1 & 0 & 0 & 0 \bigstrut \\\hline
			13             & 1 & 1 & 0 & 1 & 0 \bigstrut \\\hline
			14             & 1 & 1 & 1 & 0 & 1 \bigstrut \\\hline
			15             & 1 & 1 & 1 & 1 & 0 \bigstrut \\\hlineB{2.5}
		\end{tabular}%
	}
	\label{zad1-TableTrue}%
\end{table}%
\subsubsection{Siatka Karnaugh}
\begin{figure}[H]
	\centering
	\resizebox{0.5\textwidth}{!}{
		\begin{karnaugh-map}[4][4][1][$wx$][$yz$]
		\minterms{2,4,5,10,11} % na tych koordynatach umieść jedynki (1)
		\autoterms[0] % umieść ten symbol w miejscach niezdefiniowanych
		\implicant{4}{5} % połącz te komórki
		\implicant{11}{10}
		\implicantedge{2}{2}{10}{10} %połącz komórki brzegowe
		\end{karnaugh-map}
	}
	\caption{$ Wyj_G = w\overline{x}\overline{z} + \overline{w}\overline{y}z + wy\overline{z}$}
	\end{figure}
\subsubsection{Schemat układu}
\subsubsection{Kod VHDL}
\subsubsection{Symulacja}
\subsection{Fizyczna implementacja}
\subsubsection{Kod UCF}

\section{Zadanie 2}
\subsection{Polecenie}
Implementacja układu translatora kodu \textbf{4-bit kod NKB na 4-bit kod Aikena} w VHDL-u za pomocą:
\begin{enumerate}
	\item Zapis równań boolowskich
	\item Metoda zapisu tablicowego
\end{enumerate}
\subsection{Rozwiązanie}
\subsubsection{Schemat stanów}
\subsubsection{Tabela prawdy}
\subsubsection{Siatki Karnaugh}
\subsubsection{Schemat układu}
\subsubsection{Kod VHDL}
\subsubsection{Symulacja}
\subsection{Fizyczna implementacja}
\subsubsection{Kod UCF}

\section{Zadanie 3}
\subsection{Polecenie}
Detektor sekwencji 11011, automat Mealy-ego, jedno wejście, jedno wyjście, brak resetu, sekwencja prawidłowa 5-bitowa w VHDL-u jako maszyna stanów.
\subsection{Rozwiązanie}
\subsubsection{Schemat stanów}
\subsubsection{Tabela prawdy}
\subsubsection{Siatki Karnaugh}
\subsubsection{Schemat układu}
\subsubsection{Kod VHDL}
\subsubsection{Symulacja}
\subsection{Fizyczna implementacja}
\subsubsection{Kod UCF}

\section{Zadanie 4}
\subsection{Polecenie}
Zaprojektować licznik synchroniczny liczący w tył na bazie kodu Aikena w zakresie 0-6 (mod 7) jako maszyna stanów.
\subsection{Rozwiązanie}
\subsubsection{Schemat stanów}
\subsubsection{Tabela prawdy}
\subsubsection{Siatki Karnaugh}
\subsubsection{Schemat układu}
\subsubsection{Kod VHDL}
\subsubsection{Symulacja}
\subsection{Fizyczna implementacja}
\subsubsection{Kod UCF}

\section{Wnioski}
\end{document}