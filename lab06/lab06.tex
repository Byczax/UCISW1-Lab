%!TEX program = xelatex
%Template created by: Maciej Byczko
\documentclass[a4paper,12pt]{extarticle}  %typ dokumentu

% \usepackage[utf8]{inputenc} %rodzaj czcionki w dokumencie
\usepackage{geometry} %poprawienie marginesów
\usepackage{polski} %polskie znaki
\usepackage{multirow} %tabela
\usepackage{graphicx} %tabela
\usepackage{float} %poprawienie pozycji
\usepackage{fancyhdr} % header i footer
\usepackage{karnaugh-map} % rysowanie siatek karnaugh
\usepackage{hyperref} %tworzenie odnośników, \url{<url>}, \href{<file path, link>}{<text with link>} \pageref{}
\usepackage{amsmath} % Matma
\usepackage{boldline}%edytowanie grubości krawędzi w tabelach \hlineB{} \clineB{}{}
\usepackage{array}%grubsze kolumny w tabeli
\usepackage{bigstrut}
\usepackage{caption}
\usepackage{listings} %pisanie kodu w ładny sposób, begin{listings}[language=<język>]...end{listings} tak samo jak nazwa paczki
\usepackage{subcaption}

%Ustawienie paczki hyperref
\hypersetup{
     colorlinks,
     citecolor=black,
     filecolor=black,
     linkcolor=black,
     urlcolor=black
}

\definecolor{backcolour}{rgb}{0.95,0.95,0.92}
\definecolor{AO}{rgb}{0,0.5,0}
\definecolor{ZeroBlue}{rgb}{0,0.28,0.73}
\definecolor{DarkRed}{rgb}{0.85,0.16,0.16}

\lstset{
breaklines=true,
language=vhdl,
numbers=left,
tabsize=2,
numberstyle=\tiny,
backgroundcolor=\color{backcolour},
breakatwhitespace=false,
showspaces=false,                
showstringspaces=false,
showtabs=false,
commentstyle=\color{gray},
keywordstyle=\color{ZeroBlue},
% keywordstyle={[2]\color{DarkRed}},
% keywordstyle={[3]\color{ZeroBlue}},
}
\graphicspath{{pictures/}}
\geometry{margin=0.7in}
\pagestyle{fancy}

\cfoot{Strona \thepage}
\rhead{Strona \thepage}
\lhead{\typdoc}
\newcolumntype{?}{!{\vrule width 1.5pt}}

\title{\tytul}
\author{\tworcy}
\date{\data}

%-----------------------PRZYDATNE LINKI----------------------------------
%link do tworzenia tabeli https://tablesgenerator.com
%symbole matematyczne: https://oeis.org/wiki/List_of_LaTeX_mathematical_symbols
%narzędzia matematyczne: https://en.wikibooks.org/wiki/LaTeX/Mathematics
%krótkie podpowiedzi http://www.mif.pg.gda.pl/homepages/sylas/students/wdi/doc/latex-sciaga.html
%symbole do schematów: http://texdoc.net/texmf-dist/doc/latex/circuitikz/circuitikzmanual.pdf
%----------------------------------------------------------------------

%-----------------------SEKCJA DANYCH----------------------------------
\def\tytul{Licznik synchroniczny sterowany - FPGA} %<<< tytuł ćwiczenia
\def\nrcw{7} %<<< numer ćwiczenia
\def\data{10 Stycznia 2022r.} %<< data wykonania
\def\prowadzacy{dr inż. Jacek Mazurkiewicz} %<<<prowadzący
\def\nrgrupy{B} %<<<numer grupy
\def\tworcy{Maciej Byczko\\Bartosz Matysiak} %<<< autorzy
\def\zajinfo{PN 10:50 TP} %<<< informacje dotyczące zajęć
\def\typdoc{Sprawozdanie} %<<< typ dokumentu tj Sprawozdanie, zadania itp. {Matematyka dyskretna/Sprawozdanie z Miernictwa}
\begin{document}
\setlength{\headheight}{15pt}

\newcommand{\ov}[1]{\overline{#1} \ }

%-------------------------------------TABELA-DANYCH--------------------------------------------------
\begin{table}[H]
	\centering
	\resizebox{\textwidth}{!}{
		\begin{tabular}{|c|c|c|}\hline
			\begin{tabular}[c]{@{}c@{}}                     \tworcy     \end{tabular} &
			\begin{tabular}[c]{@{}c@{}}Prowadzący:\\        \prowadzacy \end{tabular} &
			\begin{tabular}[c]{@{}c@{}}Numer ćwiczenia\\    \nrcw       \end{tabular}          \\ \hline
			\begin{tabular}[c]{@{}c@{}}                     \zajinfo    \end{tabular} &
			\begin{tabular}[c]{@{}c@{}}Temat ćwiczenia:\\   \tytul      \end{tabular} & Ocena: \\ \hline
			\begin{tabular}[c]{@{}c@{}}Grupa:\\          \nrgrupy    \end{tabular}    &
			\begin{tabular}[c]{@{}c@{}}Data wykonania:\\    \data       \end{tabular} &        \\ \hline
		\end{tabular}}
\end{table}
%----------------------------------------------------------------------------------------------------
\tableofcontents
\cleardoublepage
% \section{Zadanie 1}
\section{Polecenie}
Licznik synchroniczny rewersyjny 8-bitowy pracujący w kodzie naturalnym binarnym. Wartość inicjująca licznik ma być ładowana z klawiatury komputera PC poprzez uruchomiony na nim terminal. Można także użyć klawiatury PS/2 - uwaga na inne wartości podawane przez klawiaturę - kody skaningowe - oraz inny moduł wejściowy do obsługi portu PS/2.

Oznacza to, że do przystawki dotrze kod naciśniętego klawisza poprzez port szeregowy RS232 lub port PS/2 i ten właśnie kod ma inicjować licznik. Licznik po przyjęciu kodu zaczyna liczyć - grupa wybiera czy będzie zwiększał swój stan - będzie początkowo - pozytywny, czy też zmniejszał swój stan - będzie początkowo - negatywny.

Bieżący stan licznika ma być wyświetlany na wyświetlaczu LCD w dowolnej, ale jednoznacznej i komunikatywnej formie. W dowolnym momencie pracy licznika możemy zmieniać kierunek zliczania wybranym guzikiem z przystawki. Może to być jeden guzik - przełącznik góra/dół, mogą być użyte dwa guziki - jeden włącza zliczanie w górę, drugi - zliczanie w dół.

Realizacja zadania wymaga zatem napisania w VHDL-u własnego modułu, który będzie realizował działanie licznika oraz spięcie tego modułu z gotowymi modułami obsługi urządzeń wejścia/wyjścia przystawki: odbiornika portu RS232 lub portu PS/2 oraz obsługi wyświetlacza LCD stanowiącego integralną część przystawki Spartan FPGA. 
\section{Rozwiązanie}
Projekt licznika się nie zmienia względem poprzednich zajęć, więc dla przypomnienia:

aby wykonać w pełni działający licznik wiemy że potrzebujemy następujące wejścia/wyjścia:
\begin{itemize}
	\item Wejście na parametry podane z klawiatury
	\item Zegar na podstawie którego wywołamy kolejny stan
	\item Reset za pomocą którego będziemy informować licznik że chcemy wprowadzić nową wartość
	\item Kontrolę w jaki sposób licznik będzie liczyć (do przodu/do tyłu)
\end{itemize}
Licznik bazuje na 8 bitach więc w momencie przepełnienia ustawiliśmy że licznik wraca do wartości:
\begin{itemize}
	\item Zerowej gdy liczy do przodu
	\item Maksymalnej gdy liczy do tyłu
\end{itemize}
Jedyna różnica w kodzie jest taka, że Spartan posiada zegar ze znacznie większą częstotliwością, więc na podstawie wykonania działania:

$$1Mhz = 10^6 Hz \rightarrow 50Mhz = 5*10^7Hz$$

$$\log_2(5*10^7Hz)\approx25.57$$

$$\frac{5*10^7Hz}{2^{25}}\approx1.49Hz$$

Dzięki tym obliczeniom wiemy, że musimy zastosować 25 bitowy dzielnik aby uzyskać częstotliwość w przybliżeniu $1Hz$.
\subsection{Kod VHDL}
\lstinputlisting{zadanie1/mainModule.vhd}
\subsection{Kod VHDL TestBench}
\lstinputlisting{zadanie1/counter_testbench.vhd}

\subsection{Symulacja}
\begin{figure}[H]
   \centering
   \caption{Początek symulacji}
   \resizebox*{\textwidth}{!}{
	  \includegraphics{zadanie1/simulation_run_1.png}
   }
\end{figure}
\begin{figure}[H]
	\centering
	\caption{Wprowadzenie nowej wartości}
	\resizebox*{\textwidth}{!}{
	   \includegraphics{zadanie1/simulation_run_2.png}
	}
 \end{figure}
 \begin{figure}[H]
	\centering
	\caption{Odwrócenie kolejności odliczania}
	\resizebox*{\textwidth}{!}{
	   \includegraphics{zadanie1/simulation_run_3.png}
	}
 \end{figure}
 \subsection{Schemat układu z wykorzystaniem zaprojektowanego modułu}
 \begin{figure}[H]
	 \centering
	 \caption{Schemat z podłączoną klawiaturą oraz wyświetlaczem LED}
	 \resizebox*{\textwidth}{!}{
		\includegraphics{zadanie1/scheme.png}
	 }
  \end{figure}
\section{Fizyczna implementacja}
\subsection{Kod UCF}
Normalnie Kod byłby w dwóch plikach:
GenIO.ucf oraz LDC.ucf lecz w celu poprawienia czytelności kody zostały umieszczone w jednym bloku 
\lstinputlisting{zadanie1/Combined.ucf}

\section{Wnioski}
Niestety przez zajęcia zdalne nie mieliśmy możliwości przetestowania zaprojektowanego układu.


\end{document}