%!TEX program = xelatex
%Template created by: Maciej Byczko
\documentclass[a4paper,12pt]{extarticle}  %typ dokumentu

% \usepackage[utf8]{inputenc} %rodzaj czcionki w dokumencie
\usepackage{geometry} %poprawienie marginesów
\usepackage{polski} %polskie znaki
\usepackage{multirow} %tabela
\usepackage{graphicx} %tabela
\usepackage{float} %poprawienie pozycji
\usepackage{fancyhdr} % header i footer
\usepackage{karnaugh-map} % rysowanie siatek karnaugh
\usepackage{hyperref} %tworzenie odnośników, \url{<url>}, \href{<file path, link>}{<text with link>} \pageref{}
\usepackage{amsmath} % Matma
\usepackage{boldline}%edytowanie grubości krawędzi w tabelach \hlineB{} \clineB{}{}
\usepackage{array}%grubsze kolumny w tabeli
\usepackage{bigstrut}
\usepackage{caption}
\usepackage{listings} %pisanie kodu w ładny sposób, begin{listings}[language=<język>]...end{listings} tak samo jak nazwa paczki
\usepackage{subcaption}

%Ustawienie paczki hyperref
\hypersetup{
     colorlinks,
     citecolor=black,
     filecolor=black,
     linkcolor=black,
     urlcolor=black
}

\definecolor{backcolour}{rgb}{0.95,0.95,0.92}
\definecolor{AO}{rgb}{0,0.5,0}
\definecolor{ZeroBlue}{rgb}{0,0.28,0.73}
\definecolor{DarkRed}{rgb}{0.85,0.16,0.16}

\lstset{
breaklines=true,
language=vhdl,
numbers=left,
tabsize=2,
numberstyle=\tiny,
backgroundcolor=\color{backcolour},
breakatwhitespace=false,
showspaces=false,                
showstringspaces=false,
showtabs=false,
commentstyle=\color{gray},
keywordstyle=\color{ZeroBlue},
% keywordstyle={[2]\color{DarkRed}},
% keywordstyle={[3]\color{ZeroBlue}},
}
\graphicspath{{pictures/}}
\geometry{margin=0.7in}
\pagestyle{fancy}

\cfoot{Strona \thepage}
\rhead{Strona \thepage}
\lhead{\typdoc}
\newcolumntype{?}{!{\vrule width 1.5pt}}

\title{\tytul}
\author{\tworcy}
\date{\data}

%-----------------------PRZYDATNE LINKI----------------------------------
%link do tworzenia tabeli https://tablesgenerator.com
%symbole matematyczne: https://oeis.org/wiki/List_of_LaTeX_mathematical_symbols
%narzędzia matematyczne: https://en.wikibooks.org/wiki/LaTeX/Mathematics
%krótkie podpowiedzi http://www.mif.pg.gda.pl/homepages/sylas/students/wdi/doc/latex-sciaga.html
%symbole do schematów: http://texdoc.net/texmf-dist/doc/latex/circuitikz/circuitikzmanual.pdf
%----------------------------------------------------------------------

%-----------------------SEKCJA DANYCH----------------------------------
\def\tytul{Układy Sekwencyjne} %<<< tytuł ćwiczenia
\def\nrcw{2} %<<< numer ćwiczenia
\def\data{10 Października 2021} %<< data wykonania
\def\prowadzacy{dr inż. Jacek Mazurkiewicz} %<<<prowadzący
\def\nrgrupy{B} %<<<numer grupy
\def\tworcy{Maciej Byczko\\Bartosz Matysiak} %<<< autorzy
\def\zajinfo{PN 10:50 TP} %<<< informacje dotyczące zajęć
\def\typdoc{Sprawozdanie} %<<< typ dokumentu tj Sprawozdanie, zadania itp. {Matematyka dyskretna/Sprawozdanie z Miernictwa}
\begin{document}
\setlength{\headheight}{15pt}

\newcommand{\ov}[1]{\overline{#1} \ }

%-------------------------------------TABELA-DANYCH--------------------------------------------------
\begin{table}[H]
	\centering
	\resizebox{\textwidth}{!}{
		\begin{tabular}{|c|c|c|}\hline
			\begin{tabular}[c]{@{}c@{}}                     \tworcy     \end{tabular} &
			\begin{tabular}[c]{@{}c@{}}Prowadzący:\\        \prowadzacy \end{tabular} &
			\begin{tabular}[c]{@{}c@{}}Numer ćwiczenia\\    \nrcw       \end{tabular}          \\ \hline
			\begin{tabular}[c]{@{}c@{}}                     \zajinfo    \end{tabular} &
			\begin{tabular}[c]{@{}c@{}}Temat ćwiczenia:\\   \tytul      \end{tabular} & Ocena: \\ \hline
			\begin{tabular}[c]{@{}c@{}}Grupa:\\          \nrgrupy    \end{tabular}    &
			\begin{tabular}[c]{@{}c@{}}Data wykonania:\\    \data       \end{tabular} &        \\ \hline
		\end{tabular}}
\end{table}
%----------------------------------------------------------------------------------------------------
\tableofcontents
\cleardoublepage
\section{Zadanie 1}
\subsection{Polecenie}
Zaprojektować licznik synchroniczny liczący w tył na bazie kodu Aikena w zakresie 0-6 (mod 7).
% Licznik synchroniczny, mod 7, negatywny, kod Aikena
\subsection{Rozwiązanie}
\subsubsection{Schemat stanów}
\begin{figure}[H]
	\centering
	\resizebox*{0.65\textwidth}{!}{
		\includegraphics{zadanie1/stany.pdf}
	}
\end{figure}
\subsubsection{Tabela prawdy}
% Table generated by Excel2LaTeX from sheet 'Sheet1'
\begin{table}[H]
	\centering
	% \caption{Add caption}
	% \centering
	\resizebox*{\textwidth}{!}{
		\begin{tabular}{|c?c|c|c|c?c|c|c|c?c|c|c|c|c|c|c|c|}
			\hline
			\multirow{2}[3]{*}{n} & \multicolumn{4}{c?}{Q(t)} & \multicolumn{4}{c?}{Q(t+1)} & \multicolumn{8}{c|}{JK} \bigstrut                                                                                                                    \\
			\clineB{2-17}{2.5}    & $Q_3$                     & $Q_2$                       & $Q_1$                             & $Q_0$ & $Q_3$ & $Q_2$ & $Q_1$ & $Q_0$ & $J_3$ & $K_3$ & $J_2$ & $K_2$ & $J_1 $ & $K_1$ & $J_0$ & $K_0$ \bigstrut \\
			\hlineB{2.5}
			0                     & 0                         & 0                           & 0                                 & 0     & 1     & 1     & 0     & 0     & 1     & -     & 1     & -     & 0      & -     & 0     & - \bigstrut     \\
			\hline
			1                     & 0                         & 0                           & 0                                 & 1     & 0     & 0     & 0     & 0     & 0     & -     & 0     & -     & 0      & -     & -     & 1 \bigstrut     \\
			\hline
			2                     & 0                         & 0                           & 1                                 & 0     & 0     & 0     & 0     & 1     & 0     & -     & 0     & -     & -      & 1     & 1     & - \bigstrut     \\
			\hline
			3                     & 0                         & 0                           & 1                                 & 1     & 0     & 0     & 1     & 0     & 0     & -     & 0     & -     & -      & 0     & -     & 1 \bigstrut     \\
			\hline
			4                     & 0                         & 1                           & 0                                 & 0     & 0     & 0     & 1     & 1     & 0     & -     & -     & 1     & 1      & -     & 1     & - \bigstrut     \\
			\hline
			5                     & 1                         & 0                           & 1                                 & 1     & 0     & 1     & 0     & 0     & -     & 1     & 1     & -     & -      & 1     & -     & 1 \bigstrut     \\
			\hline
			6                     & 1                         & 1                           & 0                                 & 0     & 1     & 0     & 1     & 1     & -     & 0     & -     & 1     & 1      & -     & 1     & - \bigstrut     \\
			\hline
		\end{tabular}%
	}
	\label{tab:states}%
\end{table}%
\subsubsection{Siatki Karnaugh}
\begin{figure}[H]
\centering
\begin{minipage}[c]{0.49\linewidth}
\begin{karnaugh-map}[4][4][1][$Q_1Q_0$][$Q_3Q_2$]
\minterms{0} % na tych koordynatach umieść jedynki (1)
\maxterms{1,2,3,4}
\autoterms[-] % umieść ten symbol w miejscach niezdefiniowanych
\implicantedge{0}{0}{8}{8}
\end{karnaugh-map}
\centering
\vspace{-1cm}
\hspace{1cm}
\caption*{$J_3 = \ov{Q_2}\ov{Q_1}\ov{Q_0}$}
\end{minipage}
\centering
\begin{minipage}[c]{0.49\linewidth}
\begin{karnaugh-map}[4][4][1][$Q_1Q_0$][$Q_3Q_2$]
\minterms{0,11} % na tych koordynatach umieść jedynki (1)
\maxterms{1,2,3}
\autoterms[-] % umieść ten symbol w miejscach niezdefiniowanych
\implicant{0}{8}
\implicant{12}{10}
\end{karnaugh-map}
\centering
\vspace{-1cm}
\hspace{1cm}
\caption*{$J_2 = \ov{Q_1}\ov{Q_0} + Q_3$}
\end{minipage}
\end{figure}
\vspace{-1cm}
\begin{figure}[H]
\centering
\begin{minipage}[c]{0.49\linewidth}
\begin{karnaugh-map}[4][4][1][$Q_1Q_0$][$Q_3Q_2$]
\minterms{4,12} % na tych koordynatach umieść jedynki (1)
\maxterms{0,1}
\autoterms[-] % umieść ten symbol w miejscach niezdefiniowanych
\implicant{4}{14}
\end{karnaugh-map}
\centering
\vspace{-1cm}
\hspace{1cm}
\caption*{$J_1 = Q_2$}
\end{minipage}
\centering
\begin{minipage}[c]{0.49\linewidth}
\begin{karnaugh-map}[4][4][1][$Q_1Q_0$][$Q_3Q_2$]
\minterms{2,4,12} % na tych koordynatach umieść jedynki (1)
\maxterms{0}
\autoterms[-] % umieść ten symbol w miejscach niezdefiniowanych
\implicant{4}{14}
\implicant{3}{10}
\end{karnaugh-map}
\centering
\vspace{-1cm}
\hspace{1cm}
\caption*{$J_0 = Q_1 + Q_2$}
\end{minipage}
\vspace{-0.1cm}
\end{figure}
\vspace{-1cm}
%? Siatki dla K
\begin{figure}[H]
\centering
\begin{minipage}[c]{0.49\linewidth}
\begin{karnaugh-map}[4][4][1][$Q_1Q_0$][$Q_3Q_2$]
\minterms{11} % na tych koordynatach umieść jedynki (1)
\maxterms{12}
\autoterms[-] % umieść ten symbol w miejscach niezdefiniowanych
% \implicantedge{0}{0}{8}{8}
\implicant{3}{10}
%\implicant{1}{11}
%\implicantedge{0}{2}{8}{10}
\end{karnaugh-map}
\centering
\vspace{-1cm}
\hspace{1cm}
\caption*{$K_3 = Q_1$} %* Wybierz jedno
\end{minipage}
\centering
\begin{minipage}[c]{0.49\linewidth}
\begin{karnaugh-map}[4][4][1][$Q_1Q_0$][$Q_3Q_2$]
\minterms{4,12} % na tych koordynatach umieść jedynki (1)
\maxterms{}
\autoterms[-] % umieść ten symbol w miejscach niezdefiniowanych
\implicant{0}{10}
% \implicant{12}{10}
\end{karnaugh-map}
\centering
\vspace{-1cm}
\hspace{1cm}
\caption*{$K_2 = 1$}
\end{minipage}
\end{figure}
\vspace{-1cm}
\begin{figure}[H]
\centering
\begin{minipage}[c]{0.49\linewidth}
\begin{karnaugh-map}[4][4][1][$Q_1Q_0$][$Q_3Q_2$]
\minterms{2,11} % na tych koordynatach umieść jedynki (1)
\maxterms{3}
\autoterms[-] % umieść ten symbol w miejscach niezdefiniowanych
% \implicant{4}{14}
\implicantedge{0}{8}{2}{10}
\implicant{12}{10}
\end{karnaugh-map}
\centering
\vspace{-1cm}
\hspace{1cm}
\caption*{$K_1 = \ov{Q_0} + Q_3$}
\end{minipage}
\centering
\begin{minipage}[c]{0.49\linewidth}
\begin{karnaugh-map}[4][4][1][$Q_1Q_0$][$Q_3Q_2$]
\minterms{1,3,11} % na tych koordynatach umieść jedynki (1)
% \maxterms{}
\autoterms[-] % umieść ten symbol w miejscach niezdefiniowanych
% \implicant{4}{14}
% \implicant{3}{10}
\implicant{0}{10}
\end{karnaugh-map}
\centering
\vspace{-1cm}
\hspace{1cm}
\caption*{$K_0 = 1$}
\end{minipage}
\end{figure}
\subsubsection{Schemat układu}
% \begin{figure}[H]
% 	\centering
% 	\resizebox*{\textwidth}{!}{
% 		\includegraphics{zadanie1/schematic_zad1.png}
% 	}
% \end{figure}
\subsubsection{Kod VHDL}
% \lstinputlisting{zadanie1/vhdl_zad1.vhd}
\subsubsection{Symulacja}
% \begin{figure}[H]
% 	\centering
% 	\resizebox*{\textwidth}{!}{
% 		\includegraphics{zadanie1/simulation_zad1.png}
% 	}
% \end{figure}
\section{Zadanie 2}
\subsection{Polecenie}
Detektor sekwencji 11011, automat Mealy-ego, jedno wejście, jedno wyjście, brak resetu, sekwencja prawidłowa 5-bitowa.
\subsection{Rozwiązanie}
\subsubsection{Opis symboliki}
\textbf{Alfabet wejściowy}
\begin{itemize}
    \item $z_0 = 0$
    \item $z_1 = 1$
\end{itemize}
\textbf{Stany wewnętrzne}
\begin{itemize}
    \item $q_0$ - stan początkowy | wprowadzono niepoprawny ciąg bitów
    \item $q_1$ - wprowadzono pierwszą cyfrę prawidłowego ciągu
    \item $q_2$ - wprowadzono drugą cyfrę prawidłowego ciągu
    \item $q_3$ - wprowadzono trzecią cyfrę prawidłowego ciągu
    \item $q_4$ - wprowadzono czwartą cyfrę prawidłowego ciągu
    \item $q_5$ - wprowadzono poprawną sekwencję
\end{itemize}
\textbf{Alfabet wyjścia}
\begin{itemize}
    \item $y_0$ - Wprowadzony ciąg nadal jest niepoprawny
    \item $y_1$ - Wprowadzono poprawną sekwencję
\end{itemize}
\subsubsection{Schemat grafowy}
\begin{figure}[H]
	\centering
	\resizebox*{\textwidth}{!}{
		\includegraphics{zadanie2/mealy.pdf}
	}
\end{figure}
\subsubsection{Tabela prawdy}
% Table generated by Excel2LaTeX from sheet 'Sheet1'
% Table generated by Excel2LaTeX from sheet 'Sheet1'
\begin{table}[H]
    \centering
    % \caption{Add caption}
    \resizebox*{\textwidth}{!}{
      \begin{tabular}{|c?c|c|c?c?c|c|c?c?c|c|c|}
      \hline
      \multirow{2}[4]{*}{S} & \multicolumn{3}{c?}{Q(t)} & \multirow{2}[4]{*}{Z} & \multicolumn{3}{c?}{Q(t+1)} & \multirow{2}[4]{*}{Y} & \multicolumn{3}{c|}{D(t)} \bigstrut\\
  \clineB{2-4}{2.5}\clineB{6-8}{2.5}\clineB{10-12}{2.5}      & $Q_2$ & $Q_1$ & $Q_0$ &   & $Q_2$ & $Q_1$ & $Q_0$ &   & $T_2$ & $T_1$ & $T_0$ \bigstrut\\
      \hlineB{2.5}
      $Q_0$ & 0 & 0 & 0 & 0 & 0 & 0 & 0 & 0 & 0 & 0 & 0 \bigstrut\\
      \hline
      $Q_0$ & 0 & 0 & 0 & 1 & 0 & 0 & 1 & 0 & 0 & 0 & 1 \bigstrut\\
      \hline
      $Q_1$ & 0 & 0 & 1 & 0 & 0 & 0 & 0 & 0 & 0 & 0 & 1 \bigstrut\\
      \hline
      $Q_1$ & 0 & 0 & 1 & 1 & 0 & 1 & 0 & 0 & 0 & 1 & 1 \bigstrut\\
      \hline
      $Q_2$ & 0 & 1 & 0 & 0 & 0 & 1 & 1 & 0 & 0 & 0 & 1 \bigstrut\\
      \hline
      $Q_2$ & 0 & 1 & 0 & 1 & 0 & 1 & 0 & 0 & 0 & 0 & 0 \bigstrut\\
      \hline
      $Q_3$ & 0 & 1 & 1 & 0 & 0 & 0 & 0 & 0 & 0 & 1 & 1 \bigstrut\\
      \hline
      $Q_3$ & 0 & 1 & 1 & 1 & 1 & 0 & 0 & 0 & 1 & 1 & 1 \bigstrut\\
      \hline
      $Q_4$ & 1 & 0 & 0 & 0 & 0 & 0 & 0 & 0 & 1 & 0 & 0 \bigstrut\\
      \hline
      $Q_4$ & 1 & 0 & 0 & 1 & 1 & 0 & 1 & 0 & 0 & 0 & 1 \bigstrut\\
      \hline
      $Q_5$ & 1 & 0 & 1 & 0 & 0 & 1 & 1 & 1 & 1 & 1 & 0 \bigstrut\\
      \hline
      $Q_5$ & 1 & 0 & 1 & 1 & 0 & 1 & 0 & 1 & 1 & 1 & 1 \bigstrut\\
      \hline
      - & 1 & 1 & 0 & 0 & - & - & - & - & - & - & - \bigstrut\\
      \hline
      - & 1 & 1 & 0 & 1 & - & - & - & - & - & - & - \bigstrut\\
      \hline
      - & 1 & 1 & 1 & 0 & - & - & - & - & - & - & - \bigstrut\\
      \hline
      - & 1 & 1 & 1 & 1 & - & - & - & - & - & - & - \bigstrut\\
      \hline
      \end{tabular}%
    }
    \label{tab:mealy}%
  \end{table}%
  
\subsubsection{Siatka Karnaugh}
\begin{figure}[H]
    \centering
    \begin{minipage}[c]{0.49\linewidth}
    \begin{karnaugh-map}[4][4][1][$Q_0Z$][$Q_2Q_1$]
    \minterms{7,8,10,11}
    \maxterms{0,1,2,3,4,5,6,9}
    \autoterms[-] % umieść ten symbol w miejscach niezdefiniowanych
    \implicant{7}{15}
    \implicantedge{12}{8}{14}{10}
    \implicant{15}{10}
    % \minterms{0} % na tych koordynatach umieść jedynki (1)
    % \maxterms{1,2,3,4}
    % \implicantedge{0}{0}{8}{8}
    \end{karnaugh-map}
    \centering
    \vspace{-1cm}
    \hspace{1cm}
    \caption*{$T_2 = Q_1Q_0Z + Q_2\ov{Z} + Q_2Q_0$}
    \end{minipage}
    \centering
    \begin{minipage}[c]{0.49\linewidth}
    \begin{karnaugh-map}[4][4][1][$Q_0Z$][$Q_2Q_1$]
    \minterms{3,6,7,10,11}
    \maxterms{0,1,2,4,5,8,9}
    \autoterms[-] % umieść ten symbol w miejscach niezdefiniowanych
    \implicant{3}{11}
    \implicant{7}{14}
    \implicant{15}{10}
    % \minterms{0,11} % na tych koordynatach umieść jedynki (1)
    % \maxterms{1,2,3}
    % \implicant{0}{8}
    % \implicant{12}{10}
    \end{karnaugh-map}
    \centering
    \vspace{-1cm}
    \hspace{1cm}
    \caption*{$T_1 = Q_0Z + Q_1Q_0 + Q_2Q_0$}
    \end{minipage}
    \end{figure}
    \vspace{-1cm}
    \begin{figure}[H]
    \centering
    \begin{minipage}[c]{0.49\linewidth}
    \begin{karnaugh-map}[4][4][1][$Q_0Z$][$Q_2Q_1$]
    \minterms{1,2,3,4,6,7,9,11}
    \maxterms{0,5,8,10}
    \implicant{3}{6}
    \implicantedge{4}{12}{6}{14}
    \implicantedge{1}{3}{9}{11}
    % \minterms{4,12} % na tych koordynatach umieść jedynki (1)
    % \maxterms{0,1}
    \autoterms[-] % umieść ten symbol w miejscach niezdefiniowanych
    % \implicant{4}{14}
    \end{karnaugh-map}
    \centering
    \vspace{-1cm}
    \hspace{1cm}
    \caption*{$T_0 = \ov{Q_2}Q_0 + Q_1\ov{Z} + \ov{Q_1}Z$}
    \end{minipage}
    \begin{minipage}[c]{0.49\linewidth}
        \begin{karnaugh-map}[4][4][1][$Q_0Z$][$Q_2Q_1$]
        \minterms{10,11}
        \maxterms{0,1,2,3,4,5,6,7,8,9}
        \implicant{15}{10}
        % \minterms{4,12} % na tych koordynatach umieść jedynki (1)
        % \maxterms{0,1}
        \autoterms[-] % umieść ten symbol w miejscach niezdefiniowanych
        % \implicant{4}{14}
        \end{karnaugh-map}
        \centering
        \vspace{-1cm}
        \hspace{1cm}
        \caption*{$Y = Q_2Q_0 $}
        \end{minipage}
    % \centering
    % \begin{minipage}[c]{0.49\linewidth}
    % \begin{karnaugh-map}[4][4][1][$Q_1Q_0$][$Q_3Q_2$]
    % \minterms{2,4,12} % na tych koordynatach umieść jedynki (1)
    % \maxterms{0}
    % \autoterms[-] % umieść ten symbol w miejscach niezdefiniowanych
    % \implicant{4}{14}
    % \implicant{3}{10}
    % \end{karnaugh-map}
    % \centering
    % \vspace{-1cm}
    % \hspace{1cm}
    % \caption*{$J_0 = Q_1 + Q_2$}
    % \end{minipage}
    % \vspace{-0.1cm}
    \end{figure}

\subsubsection{Schemat układu}
\subsubsection{Kod VHDL}
\subsubsection{Symulacja}

\section{Wnioski}
\end{document}